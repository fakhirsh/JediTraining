\chapter{Introduction}

\section*{Exercise Solutions}

\begin{enumerate}
	\item Bounded by one plane: \textbf{half-space} \\
	Bounded by two planes: \textbf{infinite wedge??} \\
	Bounded by three planes: \textbf{open ended infinite triangular prism??} \\
	Bounded by four planes: \textbf{tetrahedron}
	
	\item Suppose figure \textit{A} is congruent to \textit{B} and figure \textit{B} is congruent to \textit{C}. Align \textit{A} and \textit{B} such that they completely coincide. We essentially have two names \textit{A} and \textit{B} for the same geometric figure. Now \textit{C} can be super-imposed onto \textit{B}, which is the same as \textit{A}. So \textit{C} coincides completely with \textit{A} hence \textit{C} and \textit{A} are congruent as well.
	
	\item If two different lines don't intersect they meet at zero points. If two lines intersect at two points then they intersect at all other points. There can only be one unique line that passes through two given points, this means that the two lines are essentially identical. The only case left where two different lines can intersect at one point.
	
	\item Pick any two points on the plane. If a line passes through these two points then ALL of its points lie on the plane as well. If the line does not intersect with any other line lying on the plane then the line itself does not intersect with the plane. The only case left where a line can intersect with the plane at only ONE point.
	
	\item 
	
	\item
	
	\item Suppose the line contains two points \textit{A} and \textit{B}. Now pick another point \textit{C} on the line in the direction of \textit{AB}. If \textit{B} lines on \textit{AC} then the line is straight. This condition must hold true for any given point \textit{C} on the line.
	
	\item 
	
	\item For any point \textit{A} on the plane, pick another point \textit{B} also on the plane. In a proof above we say that there exists a line on the plane passing through two points on the plane. Fixing \textit{A} you can pick \textit{B} in infinite ways. Hence there are infinite number of lines that pass through any given point \textit{A} on the plane.
	
	\item Cylinder, saddle etc
	
	\item \underline{Infinite lines}: Since the lines are infinite, it is possible to find two congruent segments (one on each line). Now align the lines such that the two segments super-impose end to end. We know that if two lines intersect at two points then they intersect at all points. Hence infinite lines are congruent.
	
	\underline{Rays}: Find two congruent segments (one on each ray such that one of the end points of the segments is the end point of the ray). Apply the argument given above.
	
	\item 
	
	\item Yes, it is unique. First find a sum of the given segments to obtain segment \textit{A}. Now find a segment on an infinite line a segment (not containing \textit{A}) that is congruent to \textit{A}. Let's call it \textit{B}. It is easy to show that \textit{B} is also the sum of given smaller segments because that's how we constructed \textit{B}, an identical copy of \textit{A}.
	
	\item yes, yes, yes
	
	\item Many examples. No, that is impossible.
	
	\item 
	
	\item 
	
	\item Couldn't understand the statement
	
	\item $\times$
	
\end{enumerate}