\documentclass[]{article}

%opening
\title{Session 1}
\author{Fakhir}

\begin{document}

\maketitle

\section*{Solutions}

\begin{enumerate}
	\item \begin{enumerate}
		\item  \begin{enumerate}
			\item 0.53125
			\item 0.15625
			\item -0.41
			\item -0.59375
			\end{enumerate}
		\item -0.16
		\item $\Delta x \rightarrow \left\lbrace -0.18, 0.10\right\rbrace $
		\end{enumerate}

	\item \begin{enumerate}
		\item  \begin{enumerate}
			\item -0.88
			\item -0.97
			\item -0.97
			\item -0.88
			\end{enumerate}
		\item -1.0
		\item $\Delta x \rightarrow \left\lbrace -0.44,0.44\right\rbrace $
		\end{enumerate}
	
	\item \begin{enumerate}
		\item  \begin{enumerate}
			\item -0.59
			\item -0.41
			\item 0.16
			\item 0.53
		\end{enumerate}
		\item -0.16
		\item $\Delta x \rightarrow \left\lbrace -0.1,0.08\right\rbrace $
	\end{enumerate}
	
	\item \begin{enumerate}
		\item No, it was a range in each case. Maybe I didn't understand the problem?? Maybe I did it correctly \textit{(just reviewed the solutions)}.
		\item When the curve tends to be pretty ``straight line-ish'' \textit{($\leftarrow$ Got this one wrong. When the curve is very ``bendy'' or ``curvy'' then $\Delta x$ has to be pretty close to $0$ in order for the slopes to be equal)}.
	\end{enumerate}

\end{enumerate}

\end{document}
