\documentclass[]{article}

\usepackage{graphicx}
\usepackage{longtable}
\usepackage{hyperref}
\usepackage{color}
\usepackage{soul}
\usepackage{amsmath}
\usepackage{amssymb}


\DeclareRobustCommand{\hlcyan}[1]{{\sethlcolor{cyan}\hl{#1}}}
\DeclareRobustCommand{\hlgreen}[1]{{\sethlcolor{green}\hl{#1}}}
\DeclareRobustCommand{\hlred}[1]{{\sethlcolor{red}\hl{#1}}}
\DeclareRobustCommand{\hlyellow}[1]{{\sethlcolor{yellow}\hl{#1}}}
\DeclareRobustCommand{\hlorange}[1]{{\sethlcolor{orange}\hl{#1}}}


%opening
\title{Session 9}
\author{Fakhir}

\begin{document}

\maketitle

\section*{Solutions}

\begin{enumerate}
	\item \begin{enumerate}
		\item For continuity $\lim_{x\to0^{-}} f(x) = \lim_{x\to0^{+}} f(x) = 6$.
		\item First derivative of the first piece must be equal to the first derivative of the second piece at 0. This means:
		$$ f'(0^{-}) = f'(0^{+})$$
		$$ 2ax + b = 10x^{4}+12x^{3}+8x+5 $$
		Putting $x = 0$ we get:
		$$b = 5$$
		\item \textit{(According to MiT solutions, no need to check this)} Second derivative of the first piece must be equal to the second derivative of the second piece at 0. This means:
		$$ f''(0^{-}) = f''(0^{+})$$
		$$ 2a = 40x^{3}+36x^{2}+8 $$
		Putting $x = 0$ we get:
		$$a = 4$$
		
		\hlred{MiT Solution says: The first derivative has to be equal on both sides. We do not need to check the second derivative. So $b=5$ and a can be any real number.}
	\end{enumerate}
	
	\item \begin{enumerate}
		\item For continuity $\lim_{x\to1^{-}} f(x) = \lim_{x\to1^{+}} f(x)$. So plugging in $x=1$ in both pieces we get:
		
		$$ a+b+6 = 20 $$
		$$ a+b = 14 $$
		\item First derivative of the first piece must be equal to the first derivative of the second piece at 1. This means:
		$$ f'(1^{-}) = f'(1^{+})$$
		$$ 2a+b = 10+12+8+5 $$
		$$ 2a+b = 35 $$
	
		Solving these two simultaneous equations we get:
		$$ 2a + (14-a) = 35 $$
		$$ a + 14 = 35 $$
		$$ a = 21 $$
		$$ b = -7 $$
	
		Hence $a=21$ and $b=-7$.
	
	\end{enumerate}
	
\end{enumerate}

\end{document}
