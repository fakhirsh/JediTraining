\documentclass[]{article}

\usepackage{graphicx}
\usepackage{longtable}
\usepackage{hyperref}
\usepackage{color}
\usepackage{soul}
\usepackage{amsmath}


\DeclareRobustCommand{\hlcyan}[1]{{\sethlcolor{cyan}\hl{#1}}}
\DeclareRobustCommand{\hlgreen}[1]{{\sethlcolor{green}\hl{#1}}}
\DeclareRobustCommand{\hlred}[1]{{\sethlcolor{red}\hl{#1}}}
\DeclareRobustCommand{\hlyellow}[1]{{\sethlcolor{yellow}\hl{#1}}}
\DeclareRobustCommand{\hlorange}[1]{{\sethlcolor{orange}\hl{#1}}}


%opening
\title{Session 5}
\author{Fakhir}

\begin{document}

\maketitle

\section*{Solutions}

\begin{enumerate}
	\item Plot $\sqrt{x}$. Observing it gives us a one sided limit i.e limit from the right. It is a continuous function (how?) but it is not defined for $x<0$:
	$$\lim_{x\to0^{+}}\sqrt{x} = 0$$
	$$\lim_{x\to0^{-}}\sqrt{x} = invalid$$ 
	
	\item Visualizing the graph gives us two sided limit. This function is not defined at $x=-1$:
	$$\lim_{x\to-1^{+}}\frac{1}{x+1} = +\infty$$
	$$\lim_{x\to-1^{-}}\frac{1}{x+1} = -\infty$$ 
	
	\item First by visualizing the graph (and then bit cheating by plotting it in maxima) gives us two sided limit. This function is not defined at $x=1$:
	$$\lim_{x\to1^{+}}\frac{1}{(x-1)^{4}} = +\infty$$
	$$\lim_{x\to1^{-}}\frac{1}{(x-1)^{4}} = +\infty$$
	
	Since in this case $\lim_{x\to1^{-}}\frac{1}{(x-1)^{4}} = \lim_{x\to1^{+}}\frac{1}{(x-1)^{4}} $ \\
	
	We can simply say:
	
	$$\lim_{x\to1}\frac{1}{(x-1)^{4}} = +\infty$$
	
	\item First by visualizing the graph (couldn't draw on maxima) gives us two sided limit. This function appears to be continuous at all points:
	$$\lim_{x\to0^{+}}|\sin{x}| = 0$$
	$$\lim_{x\to0^{-}}|\sin{x}| = 0$$
	
	Again we can say:
	
	$$\lim_{x\to0}|\sin{x}| = 0$$

	\item Couldn't plot it in maxima. In my opinion it is a piece wise graph given by:
	\[
	f(x)= 
	\begin{cases}
	1, & \text{if } x>0 \\
	-1,& \text{if } x<0 \\
	undefined, & \text{if } x=0
	\end{cases}
	\]
	
	So this implies that it has a two sided limit as follows:
	$$\lim_{x\to0^{+}}\frac{|x|}{x} = +1$$
	$$\lim_{x\to0^{-}}\frac{|x|}{x} = -1$$ \\~\\
	
\end{enumerate}

\hlred{Extra questions that need to be answered:}
\begin{enumerate}
	\item I didn't quite understand what he means by ``one sided limit'' in the MiT solutions file. In some places (Q4 and Q5) he has used ``no need to use one sided limit here because..'', not too sure what he is saying. I thought one sided limit means the special case when either $\lim_{x\to k^{+}}$ exists or $\lim_{x\to k^{-}}$ but not both. Am I missing something here?
	
	\item Show that $\sqrt{x}$ is a continuous function specially at $x=0$ (beware: it has one sided limit).
	
	\item Show that $|\sin{x}|$ is a continuous function specially at $x=0$ 
	
	
\end{enumerate}

\end{document}
