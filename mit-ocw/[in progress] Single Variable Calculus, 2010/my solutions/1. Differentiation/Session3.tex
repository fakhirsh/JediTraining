\documentclass[]{article}

\usepackage{graphicx}
\usepackage{longtable}
\usepackage{hyperref}
\usepackage{color}
\usepackage{soul}
\usepackage{amsmath}


\DeclareRobustCommand{\hlcyan}[1]{{\sethlcolor{cyan}\hl{#1}}}
\DeclareRobustCommand{\hlgreen}[1]{{\sethlcolor{green}\hl{#1}}}
\DeclareRobustCommand{\hlred}[1]{{\sethlcolor{red}\hl{#1}}}
\DeclareRobustCommand{\hlyellow}[1]{{\sethlcolor{yellow}\hl{#1}}}
\DeclareRobustCommand{\hlorange}[1]{{\sethlcolor{orange}\hl{#1}}}


%opening
\title{Session 3}
\author{Fakhir}

\begin{document}

\maketitle

\section*{Solutions}

\begin{enumerate}
	\item $f'(1) = -10+60 = 50$ \\~\\
	The balance Feb will be more as compared to Jan because the rate of change i.e. balance difference per month is $+ve$.
	\item $f'(10) = -100+60 = -40$ \\~\\
	This shows that the balance in current month was less than the balance in the previous month. So no, the account balance does not continue to increase throughout the year.
	\item The account balance will continue to increase as long as the change in balances during consecutive months remains $+ve$. Through trial and error we find: \\
	$f'(5) = +10$ \\
	$f'(6) = 0$ \\
	and \\
	$f'(7) = -10$ \\~\\
	So the balance was greatest in the month of June. \\~\\
	
\end{enumerate}

Tip from MiT solutions: Also plug-in some values for $t$ to check the actual account balance i.e $f(t)$ in the above solutions to verify the answer.

\end{document}
