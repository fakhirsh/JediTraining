\documentclass[]{article}

\usepackage{graphicx}
\usepackage{longtable}
\usepackage{hyperref}
\usepackage{color}
\usepackage{soul}
\usepackage{amsmath}


\DeclareRobustCommand{\hlcyan}[1]{{\sethlcolor{cyan}\hl{#1}}}
\DeclareRobustCommand{\hlgreen}[1]{{\sethlcolor{green}\hl{#1}}}
\DeclareRobustCommand{\hlred}[1]{{\sethlcolor{red}\hl{#1}}}
\DeclareRobustCommand{\hlyellow}[1]{{\sethlcolor{yellow}\hl{#1}}}
\DeclareRobustCommand{\hlorange}[1]{{\sethlcolor{orange}\hl{#1}}}


%opening
\title{MiT Single Variable Calculus, Fall 2010 \\ Pending Questions}
\author{Fakhir}

\begin{document}

\maketitle

\url{https://ocw.mit.edu/courses/mathematics/18-01sc-single-variable-calculus-fall-2010/}\\~\\

Questions that I had in my mind while doing this course: \\~\\

\section{Differentiation}
\subsection{Session 1, 2}
\begin{enumerate}
	\item How do we cut a cone using a plane so that it gives us the hyperbola $1/x$ (because this particular curve has a weird skewed shape. Pretty hard to imagine the resulting cross-section unless the cones are themselves skewed somehow.)
	
	\item The ``difference quotient'' formula for differentiation has delta x or dx in the denominator where $dx \rightarrow 0$. We can't just put dx to 0 ``INITIALLY'' because that will give us $0/0$. But after simplifying the formula (and some cancellations), as a very last step we plug 0 to $dx$ and everything works out magically ! So What is happening here? Is the simplification process changing anything? (even though it shouldn't)
\end{enumerate}

\subsection{Session 3}

\begin{enumerate}
	\item The geometric interpretation of derivative is that its the "slope of a line". Are there any other interpretations? 
	Yes:
	\begin{enumerate}
		\item Rate of change (average change)
		\item ...
	\end{enumerate}

	
\end{enumerate}

\subsection{Session 5}

\begin{enumerate}
	\item Stupid question: Why do we study limits in "calculus"? Shouldn't this be a part of say functions or real analysis or something?
	
	\item Is the ``limit'' a ``linear operator''? i.e: 
	$$\lim_{x\to\infty} \left(f(x)+g(x)\right) = \lim_{x\to\infty} f(x) + \lim_{x\to\infty} g(x)$$
	
	\item \textbf{My Comment:} Most of the proofs we do in math appear to be ``trial and error''. We keep on discarding things that do not work and accept things that do work (at least appear to). But once we reach at the end, the result may give an insight to something totally new, something that we were not even looking for initially, a novel insight into the reality (its like traveling to the unknown place through a worm hole or suddenly being illuminated by the divine).
	
	\item I didn't quite understand what he means by ``one sided limit'' in the MiT solutions file. In some places (Q4 and Q5) he has used ``no need to use one sided limit here because..'', not too sure what he is saying. I thought one sided limit means the special case when either $\lim_{x\to k^{+}}$ exists or $\lim_{x\to k^{-}}$ but not both. Am I missing something here?
	
	\item Show that $\sqrt{x}$ is a continuous function specially at $x=0$ (beware: it has one sided limit).
	
	\item Show that $|\sin{x}|$ is a continuous function specially at $x=0$ 
\end{enumerate}

\subsection{Session 7}
\begin{enumerate}
	\item What actually the trigonometric functions signify? i.e the $\sin$ function takes an angle and returns a real number. What does that number exactly represent? Geometric interpretation of trigonometric functions?
\end{enumerate}

\subsection{Session 8}
\begin{enumerate}
	\item \textbf{Clip 1:} He proved the limit $\lim_{\Delta x\to0}\frac{\sin(\Delta x)}{\Delta x}= 1$ by drawing just a figure and then ``conjecturing'' the conclusion. No rigorous mathematical steps that lead to the final answer. In the notes pdf he also substitutes a bunch of values for $\delta x$ in order to ``convince'' us about the answer.
	
	\item \textbf{Clip 2:} Similarly, $\lim_{\Delta x\to0}\frac{1-\cos(\Delta x)}{\Delta x}= 0$ proved by mere geometric ``intuition''. \\
	
	\hlgreen{Answer to above two points:} Just search the web online such as \href{https://math.stackexchange.com/questions/75130/how-to-prove-that-lim-limits-x-to0-frac-sin-xx-1}{here}.\\
	
	But now as opposed to the methods in previous sections he is trying to compute the limit very near to zero BUT NOT AT zero. So the behavior of both the numerator and denominator are being studied as $\theta$ gets very small BUT NOT EXACTLY ZERO. In previous methods we somehow tried to compute the limits exactly at $\Delta x\to 0$ by somehow trying to cancel out $\Delta x$. This means many varying methods to compute the limits, I'm confused !!!
	
	\item \textbf{Clip 4:} \hlyellow{Need to review this proof again. I was unable to understand how he proved the congruent $\theta$. I didn't understand either of the methods but specially didn't get that rotation/translation argument.}\\
	
	\underline{To ask:} The Geometric proofs do not appear to be as ``rigorous'' as analytical ones. Specially in this proof no matter how small you get, $\Delta x$ can become very very tiny but will never be 0. So how on earth are we deducing exact limits from approximations?
	
	

\end{enumerate}

\end{document}
