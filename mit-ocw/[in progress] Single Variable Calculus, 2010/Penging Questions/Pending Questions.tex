\documentclass[]{article}

\usepackage{graphicx}
\usepackage{longtable}
\usepackage{hyperref}
\usepackage{color}
\usepackage{soul}
\usepackage{amsmath}


\DeclareRobustCommand{\hlcyan}[1]{{\sethlcolor{cyan}\hl{#1}}}
\DeclareRobustCommand{\hlgreen}[1]{{\sethlcolor{green}\hl{#1}}}
\DeclareRobustCommand{\hlred}[1]{{\sethlcolor{red}\hl{#1}}}
\DeclareRobustCommand{\hlyellow}[1]{{\sethlcolor{yellow}\hl{#1}}}
\DeclareRobustCommand{\hlorange}[1]{{\sethlcolor{orange}\hl{#1}}}


%opening
\title{MiT Single Variable Calculus, Fall 2010 \\ Pending Questions}
\author{Fakhir}

\begin{document}

\maketitle

\url{https://ocw.mit.edu/courses/mathematics/18-01sc-single-variable-calculus-fall-2010/}\\~\\

Questions that I had in my mind while doing this course: \\~\\

\section{Differentiation}
\subsection{Session 1, 2}
\begin{enumerate}
	\item How do we cut a cone using a plane so that it gives us the hyperbola $1/x$ (because this particular curve has a weird skewed shape. Pretty hard to imagine the resulting cross-section unless the cones are themselves skewed somehow.)
	
	\item The ``difference quotient'' formula for differentiation has delta x or dx in the denominator where $dx \rightarrow 0$. We can't just put dx to 0 ``INITIALLY'' because that will give us $0/0$. But after simplifying the formula (and some cancellations), as a very last step we plug 0 to $dx$ and everything works out magically ! So What is happening here? Is the simplification process changing anything? (even though it shouldn't)
\end{enumerate}

\subsection{Session 3}

\begin{enumerate}
	\item The geometric interpretation of derivative is that its the "slope of a line". Are there any other interpretations? 
	Yes:
	\begin{enumerate}
		\item Rate of change (average change)
		\item ...
	\end{enumerate}

	\item Stupid question: Why do we study limits in "calculus"? Shouldn't this be a part of say functions or real analysis or something?
	
	\item Is the ``limit'' a ``linear operator''? i.e: 
	$$\lim_{x\to\infty} \left(f(x)+g(x)\right) = \lim_{x\to\infty} f(x) + \lim_{x\to\infty} g(x)$$
		
\end{enumerate}


\end{document}
